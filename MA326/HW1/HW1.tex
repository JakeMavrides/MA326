\documentclass[10pt,twoside]{article}
\usepackage{amsmath,amssymb,amsthm}
\usepackage{fancyhdr, lastpage}
\setlength{\voffset}{-1in}
\setlength{\topmargin}{0in}
\setlength{\headheight}{0.5in}
\setlength{\headsep}{0.25in}
\setlength{\textheight}{9.5in}
\setlength{\footskip}{0.5in}
\setlength{\hoffset}{0in}
\setlength{\oddsidemargin}{0in}
\setlength{\evensidemargin}{0in}
\setlength{\marginparsep}{0in}
\setlength{\marginparwidth}{0in}
\setlength{\textwidth}{6.5in}
\pagestyle{fancy}
\usepackage{hyperref}
\usepackage{graphicx}
\usepackage{url}%

%% Macros

%Spaces
\newcommand{\rn}{\mathbb{R}^{n}}
\newcommand{\rnn}{\mathbb{R}^{n\times n}}
\newcommand{\rmn}{\mathbb{R}^{m\times n}}
\newcommand{\cn}{\mathbb{C}^{n}}
\newcommand{\cnn}{\mathbb{C}^{n\times n}}
\newcommand{\cmn}{\mathbb{C}^{m\times n}}

%Various possibilities for inner products and norms
\newcommand{\inner}[2]{\langle #1\mid #2\rangle}
\newcommand{\norm}[2]{\left\|#1\right\|_{#2}}
\newcommand{\normf}[1]{\left\|#1\right\|_{F}}
\newcommand{\normtwo}[1]{\left\|#1\right\|_{2}}
\newcommand{\normp}[1]{\left\|#1\right\|_{p}}

%Boldface for vectors and tildes
\renewcommand{\vec}[1]{\mathbf{#1}}
\newcommand{\mat}[1]{\mathbf{#1}}
\newcommand{\B}[1]{\boldsymbol{#1}}
\newcommand{\vect}[1]{\widetilde{\boldsymbol{#1}}}
\newcommand{\matt}[1]{\widetilde{\boldsymbol{#1}}}

%Column and row equivalence
\newcommand{\roweq}{\stackrel{\text{row}}{\sim}}
\newcommand{\coleq}{\stackrel{\text{col}}{\sim}}

\newtheorem{definition}{Definition}
\newtheorem{example}{Example}
\newtheorem{fact}{Fact}
\newtheorem{remark}{Remark}

%Vector spaces
\newcommand{\rank}{\text{rank}\,}
\renewcommand{\dim}{\text{dim}\,}
\newcommand{\Span}[1]{\text{Span}\,\{#1\}}
\newcommand{\basis}[1]{\left\{ #1\right\}}

\fancyhf{}
\fancyhead[LO,RE]{MA 326}
\fancyhead[RO,LE]{Due 5 PM, Friday, January 26}
\chead{\textbf{Homework 1}}
\cfoot{}
\parindent 0in


\begin{document}

\vspace{0.2in}

\textbf{Homework Policies:}

When showing a theoretical statement or computation by hand is part of the exercise, the clarity and completeness of your arguments/steps will be as important as their correctness. 

\vspace{0.1in}


Your homework assignments can be handwritten or typeset, but all graphs should be included as images from the graphing tool that you use.

\vspace{0.1in}

The submission of homework that requires numerical work on a computer should include the following: printout of the code used to solve the problems, its inputs, and outputs. The code should be written clearly and should be commented in such a way that at least the inputs and outputs of the code are clear. The specific outputs requested by the exercise should be discussed in your write-up as needed in order to answer the questions in the problems. 

\vspace{0.1in}
The filenames for the code should be indicative of the exercise. For example, if it is a Jupyter notebook for homework 1 exercise 2, then it could be named \texttt{hw1Ex2.ipynb}.


\vspace{0.1in}


\textbf{\underline{Figures:}} 
Each required plot should have the following features:
\begin{itemize}
\item Clearly label the x- and y-axes. All the labels should have fontsize $18$. 
\item Add a title with fontsize $20$.
\item Add legends to clearly label the plots. The fontsize for the labels should be $18$. 
\item All the ticks in the figures should have fontsize $16$.
\item The markersize should be $10$ pts, and the line-widths should be $4$ pts. 
\end{itemize}


\textbf{\underline{Submission:}}

\begin{itemize}
\item Please submit your solutions in a \emph{PDF file}, together with \emph{a .zip file containing all the code needed to reproduce your results}. Mention the students with whom you discussed the homework. 

\item For the computer problems, include the printout of the code, inputs, outputs, required plots, and discussions needed to answer the questions (when appropriate).

\end{itemize}



Before working on the exercises:

\vspace{0.1in}
\textbf{Study and practice Python.} You may want to use the resources provided on the course Moodle page. The best way to learn would be by looking at existing code and writing a few programs yourself. If you are not sure how to write Python code properly or have never programmed before, please come to my office hours and discuss.

\vspace{0.2in}

\newpage
\textbf{Exercises:}

\vspace{0.1in}


\begin{enumerate}

\item (\emph{40 pts}) In this problem you will determine a cubic spline approximation that fits exactly through the five data points below:
\[
P_1(0, 2), \quad P_2(0.5,0), \quad P_3(1,1), \quad P_4(1.5, 2), \quad P_5(2, 0).
\]
The spline will be built using two cubic functions:
\[
y_1(x) = a_1 + a_2 x + a_3 x^2 + a_4 x^3,
\]
\[
y_2(x) = a_5 + a_6 x + a_7 x^2 + a_8 x^3.
\]
\begin{enumerate}
\item Formulate three linear algebraic equations ensuring that $y_1(x)$ goes exactly through the three points $P_1$, $P_2$, and $P_3$.
\item Formulate three linear algebraic equations ensuring that $y_2(x)$ goes exactly through the three points $P_3$, $P_4$, and $P_5$.
\item Formulate two linear algebraic equations ensuring that the slope and curvature (measured via the second derivative) are both continuous at the point $P_3$.
\item By assembling the equations found in parts (a) -- (c), determine the matrix $\mat{A}$ and vector $\vec{b}$ in the linear system
\[
\mat{A}\vec{a} = \vec{b}, \text{ where } \vec{a} = [a_1, a_2, a_3, a_4, a_5, a_6, a_7, a_8]^T.
\]
\item Solve the linear system in part (d) to determine the two cubic functions $y_1(x)$ and $y_2(x)$, respectively. You may solve the linear system using Python packages.
\item On the same graph, plot $y_1$ on the interval $[0,1]$, $y_2$ on the interval $[1,2]$, and the five data points $P_1, \dots, P_5$.
\end{enumerate}

%\newpage
\item (\emph{30 pts}) Consider the linear regression model 
\[
\hat{f}(\vec{x}) = \beta_0+\beta_1 x_1 + \beta_2 x_2, \quad \mbox{ where } \vec{x} = [x_1, x_2],
\]
for the given data 
\[
\vec{x}_1 = [1, 1], y_1 = 0.1;~ 
\vec{x}_2 = [-1,-1], y_2 = 5.95; ~
\vec{x}_3 = [0,1], y_3 = 0.8; \]
\[
\vec{x}_4 = [1, 0], y_4 = 2.1; ~
\vec{x}_5 = [1,2], y_5 = -1.8; ~
\vec{x}_6 = [2,1], y_6 = -1.05.
\]
\begin{enumerate}
\item Derive the normal equation for this least squares linear regression problem.
\item Set up the normal equation in Python and use it to determine the regression model.
\item On the same graph, plot the approximation $\hat{f}$ via a surface plot (using \texttt{plot\_surface} function) and plot the six data points. 
\item What is the residual sum of squares of this model?
\end{enumerate}

\newpage
\item (\emph{30 pts}) 
Let $\mat{X} \in \mathbb{R}^{n\times (m+1)}$ and $\vec{y}\in \mathbb{R}^{n}$. Consider the least squares problem with Tikhonov regularization for $\lambda > 0$,
\[ \min_{\vec{\beta}\in \mathbb{R}^{m+1}} \|\mat{X}\vec{\beta}-\vec{y}\|_2^2 + \lambda^2 \|\vec{\beta}\|_2^2. \] 
\begin{enumerate}
\item Derive the normal equations for this least squares problem.\\
{\em Hint}: Define $J(\vec{\beta})= \|\mat{X}\vec{\beta}-\vec{y}\|_2^2 + \lambda^2 \|\vec{\beta}\|_2^2.$, and consider its gradient and Hessian matrices.
\item Consider the linear regression model 
\[
\hat{f}(\vec{x}) =\beta_0+ \beta_1 x_1 + \beta_2 x_2, \quad \mbox{ where } \vec{x} = [x_1, x_2],
\]
for the given data 
\[
\vec{x}_1 = [1, 1], y_1 = 0.1;~ 
\vec{x}_2 = [-1,-1], y_2 = 5.95; ~
\vec{x}_3 = [0,1], y_3 = 0.8; \]
\[
\vec{x}_4 = [1, 0], y_4 = 2.1; ~
\vec{x}_5 = [1,2], y_5 = -1.8; ~
\vec{x}_6 = [2,1], y_6 = -1.05.
\]
Find the parameters $\beta_0$, $\beta_1$, and $\beta_2$ such that the following regularized squared error loss function with $\lambda = 1$ is minimized:
\[
\min_{\beta\in \mathbb{R}^3} \sum_{i=1}^6 \left(y_i- \beta_0 - \beta_{1}x_{i1} - \beta_{2}x_{i2}\right)^2 + \lambda^2(\beta_0^2 + \beta_1^2 + \beta_2^2).
\]
\item Repeat the process in part (b) with $\lambda = 2$.
\item What is the residual sum of squares when $\lambda = 1$ and when $\lambda=2$ from parts (b) -- (c)?
\end{enumerate}

\end{enumerate}

%%%% END %%%%
\end{document}
